\chapter{Conclusiones y trabajos futuros}

Desde el principio la idea de este proyecto ha sido la de realizar algo similar a proyectos que se comercializan pero con el menor coste posible. Esto se consigue gracias al uso de todo hardware libre posible adquiriéndolo al menor coste posible y utilizando software libre. De esta forma además podremos ayudar a más usuarios que estén interesados en realizar este proyecto por su cuenta.

Este proyecto me ha ayudado a adquirir conocimientos tanto relacionados con el mundo de la informática como pueden ser conocimientos impartidos durante el grado, pero en otras ramas distintas a la que yo he cursado como otros conocimiento que no son específicos del campo de la informática los cuales se vieron en el capítulo 2. Además de los adquiridos gracias a la realización de un proyecto grande como este como pueden ser:

\begin{itemize}
	\item\textbf{Cumplimiento de plazos establecidos:} En un primer momento la entrega de este proyecto estaba prevista para el mes de junio, pero por problemas durante algunas de las fases esos plazos no se cumplieron haciendo de esta forma imposible tener el proyecto finalizado para la fecha prevista por lo que se ha tenido que posponer para septiembre. Por lo tanto creo que esto no hubiera sucedido no tanto por establecer unos plazos poco realistas ya que ese aspecto creo que no se pensó del todo mal como por no seguir avanzando en otras tareas a realizar mientras no se avance en otra.
	\item\textbf{Documentarse bien antes de realizar diseños y pruebas:} En muchas ocasiones a la hora de realizar las pruebas de algún diseño no funciona de la manera correcta y en el caso del hardware puede deberse a cualquiera de los múltiples dispositivos hardware que forman el dispositivo completo por lo tanto es importante conocer perfectamente el funcionamiento y como se utiliza cada uno de ellos para poder detectar errores de una forma sencilla.
	\item\textbf{Documentación al día :} Es importante ir documentando todo lo que se va haciendo en un periodo de tiempo corto ya que si la realización de la documentación se pospone demasiado ya no se tendrán tan frescos los conocimientos que queremos plasmar en la documentación y en muchas ocasiones será necesario volver a revisarlo con el gasto de tiempo extra que esto conlleva.
\end{itemize}
	
\section{Trabajos futuros}
Algunas de las mejoras que se pueden realizar en este proyecto son las siguientes:

\begin{itemize}
	\item Mejorar el firmware para añadir nuevos parámetros como los kilowatios hora consumidos o la media de consumo.
	\item Introducir un método mediante el cuál se puedan observar los patrones de consumo de cada electrodoméstico permitiendo de esta forma al usuario conocer que dispositivos tiene conectados en cada momento en su hogar.
	\item Mejorar el servidor web para que le muestre al usuario más información así como incluir soporte para mostrar al usuario las mejoras anteriormente citadas.
	\item Añadir un sencillo sistema de recomendación de tarifas así como de potencia contratada que permitirá al usuario seleccionar la tarifa y potencia que mejor se adapte a sus necesidades.
	\item Monitorizar un año entero de consumo eléctrico de un hogar.
	\item Bigdata que nos permitirá aprender que consumos tiene cada aparto y en base a ese aprendizaje determinar que aparatos están consumiendo electricidad en cada momento.
	\item Establecer patrones de consumo de un usuario concreto para poder observar sus hábitos en lo que al consumo eléctrico se refiere.
\end{itemize}
